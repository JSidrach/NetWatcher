\chapter{Mantenimiento\label{cap:mantenimiento}}

Se expone a continuación el plan de mantenimiento para la aplicación desarrollada.
Dentro de las diferentes categorías de mantenimiento (perfectivo, adaptativo, preventivo y correctivo), solo entra dentro del alcance de este proyecto realizar un mantenimiento correctivo hasta máximo un mes después de entregar el producto final (al menos por parte del estudiante).
Esta decisión ha sido fundamentada en el límite de tiempo que se recomienda para la elaboración del Trabajo de Fin de Grado.
Se diagnosticarán y corregirán por tanto errores durante el periodo establecido, como respuesta a solicitudes de los usuarios (a través de la apertura de peticiones en el repositorio del código).

Se han identificado, no obstante, líneas de trabajo futuro y posibles mejoras (ver capítulo~\ref{cap:lineasDeTrabajoFuturo}) que no entrarían dentro de este tipo de mantenimiento, pero que se han considerado interesante plantear.
Por ello, se ha decidido liberar el código de la aplicación en \textit{GitHub} (\url{https://github.com/JSidrach/NetWatcher}) bajo la licencia \textit{MIT} \cite{mit}, de forma que el proyecto pueda ser continuado por otras personas.
Se ha seleccionado \textit{GitHub} como plataforma para alojar el código porque facilita la colaboración entre desarrolladores y porque el propio proyecto ya estaba alojado allí de manera privada, no requiriéndose una migración.
Se ha elegido la licencia \textit{MIT} por ser una de las menos restrictivas dentro de las de código abierto~\cite{licenses}, permitiendo que cualquiera pueda utilizar y modificar la aplicación siempre que se reconozca la autoría original.
Se espera así que en un futuro otras personas puedan ampliar el trabajo realizado.
