\chapter{Introducción}

Este Trabajo de Fin de Grado, realizado en colaboración con el grupo de investigación \textit{High-Performance Computing and Networking (HPCN)}, tiene como propósito el desarrollo de una interfaz web para la gestión de sondas de red de altas prestaciones.
En los siguientes apartados se explican la motivación y los objetivos principales del mismo, sucedidos de una descripción de la estructura del resto de la memoria.

\section{Motivación}

Una sonda de red es un dispositivo capaz de capturar tráfico de red (sonda pasiva) o de inyectarlo (sonda activa).
Este dispositivo puede ser algo tan sencillo como un ordenador convencional, en el cual se ha instalado una tarjeta \textit{Ethernet} estándar o una tarjeta a medida basada en \gls{FPGA}.
La aplicación a desarrollar se basará en una sonda de este último tipo~\cite{jfzazo}.

Hasta ahora, la única posibilidad existente es interaccionar con esta sonda desde la línea de comandos, lo que dificulta su gestión para personas sin conocimientos previos de la misma.
Otra dificultad en el control de la sonda es que existen numerosos aspectos relacionados con su funcionamiento que se manejan de forma externa, tales como el almacenamiento, clasificación y gestión de las \glspl{traza}.

Surge entonces la necesidad de una simplificación en el manejo de la sonda, que permita a usuarios no avanzados su utilización.
Para ello, se desarrollará una interfaz gráfica basada en tecnologías web, que facilitará conocer y manejar el estado de la sonda.
Además, permitirá administrar otros aspectos del sistema, como los mencionados anteriormente.

Por último, se considera interesante intentar abstraer la arquitectura de la solución propuesta, de forma que cubra un problema más general: manejar una sonda de red mediante una interfaz web.
Se considera que es una extensión natural del proyecto, ya que existen aspectos comunes a cualquier sonda de red, como el manejo de las \glspl{traza}, la monitorización del rendimiento o el \gls{front-end}.
Esto permitirá adaptar, sin demasiado esfuerzo, el proyecto realizado a otras sondas de red distintas de la seleccionada.

\section{Objetivos}

Los objetivos principales planteados en este Trabajo de Fin de Grado son los siguientes:

\begin{itemize}
  \item Crear una arquitectura base que formalice una solución para la gestión de sondas de red desde una interfaz web.
  Esta propuesta contemplará cómo estructurar los componentes de forma que sea extensible a otras sondas de red, no sólo la seleccionada.
  Así, mediante la abstracción sobre el problema dado, se pretende que gran parte de la solución propuesta sea reutilizable en proyectos similares, en los que el manejo de la sonda se realiza mediante la línea de comandos.

  \item Desarrollar una interfaz web que permita la gestión de una sonda de red de altas prestaciones.
  Esta interfaz permitirá, de manera visual, conocer el estado actual de la sonda de red y configurarla para reproducir o capturar tráfico de red.
  Además, la interfaz web permitirá gestionar las \glspl{traza} capturadas con la sonda mecionada.
  Se pretende así facilitar el control de todos los componentes que intervienen en la captura y reproducción de tráfico de red, y que no sea necesario conocer previamente el funcionamiento interno de la sonda para poder manejarla y experimentar con ella.

  \item Monitorizar el estado del servidor al que está conectada la sonda de red de altas prestaciones.
  Para ello, la aplicación mostrará al usuario estadísticas sobre el servidor relevantes en este contexto.
  Por una parte, se podrá conocer el espacio de almacenamiento disponible para guardar \glspl{traza}, para que el usuario pueda liberar espacio es caso de ser necesario, y prevenir así que la sonda deje de poder capturar tráfico de red.
  Por otro lado, si el sistema de archivos en que se almacenan las \glspl{traza} capturadas con la sonda está sobre un \gls{RAID}, se podrá conocer la velocidad de escritura global del sistema y de cada uno de los discos que integran el \gls{RAID}, ya que es un factor que puede limitar el rendimiento de la sonda.

  \item Registrar estadísticas sobre el uso de la aplicación, lo que ayudará a localizar y solucionar errores internos.
  Adicionalmente, se podrán analizar estos registros para conocer el grado de utilización de la sonda por diferentes usuarios, y plantearse entonces cómo optimizar el tiempo necesario para realizar las tareas más frecuentes.
\end{itemize}

\section{Estructura del documento}

En el capítulo~\ref{cap:estadoDelArte} se realiza un análisis del estado del arte.
Se analizan tanto los sistemas de captura y reproducción de tráfico web existentes como las interfaces de gestión y monitorización de estos sistemas, para posteriormente extraer conclusiones sobre lo estudiado.

En el capítulo~\ref{cap:defProyecto} se define la aplicación que se va a diseñar, así como la metodología seguida y las herramientas utilizadas en el proyecto.
En el capítulo~\ref{cap:requisitos} se describen los requisitos funcionales y no funcionales de la aplicación.

En el capítulo~\ref{cap:disenho} se formaliza el diseño de la aplicación a implementar, comentando la arquitectura de la aplicación y los módulos en los que se divide.
En el capítulo~\ref{cap:implementacion} se documenta la implementación de la aplicación, estructurada en dos partes bien diferenciadas: \gls{back-end} y \gls{front-end}.
En el capítulo~\ref{cap:pruebas} se explica el proceso de pruebas seguido para la verificación y validación de la aplicación construida, comprobando así el correcto funcionamiento de la misma.
En el capítulo~\ref{cap:mantenimiento} se espeficica cómo se va a realizar el mantenimiento de la aplicación.

En el capítulo~\ref{cap:conclusiones} se exponen las conclusiones finales sobre el trabajo realizado.
Por último, en el capítulo~\ref{cap:lineasDeTrabajoFuturo} se plantean posibles líneas de trabajo futuro que podrían ser abordadas con el objetivo de mejorar y ampliar diferentes aspectos de la aplicación desarrollada.
