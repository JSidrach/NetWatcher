\chapter{Introducción}

TODO: Introducción del trabajo
interfaz web para la gestión de sondas de red de altas prestaciones. HPCN. Ámbito, motivación y objetivos de este Trabajo de Fin de Grado, seguidos de una explicación sobre la estructura del documento.

Una sonda de red es simplemente un dispositivo capaz de capturar tráfico de red (sonda
pasiva) o de inyectarlo (sonda activa). Este dispositivo puede ser algo tan sencillo
como un ordenador convencional, en el que se ha instalado una tarjeta Ethernet
estándar o una tarjeta a medida basada en FPGA (ver la propuesta de proyecto
“Sistema basado en FPGA para la captura de tráfico en redes multigigabit Ethernet”).
Este ordenador típicamente correrá un sistema operativo Linux/GNU, y se habrán
instalado unos drivers especiales para poder acceder lo más eficientemente a la tarjeta
de red. Lo habitual es manejar la sonda desde línea de comandos. En este proyecto se
propone hacer una interfaz de usuario mucho más amigable, basada en web. En la
sonda correrá un servidor web, que mostrará una página con la que se podrá configurar
y manejar todos los aspectos de la sonda (capturar tráfico, reproducirlo, estado de la
sonda). Todas estas operaciones se corresponden con ejecutar programas de línea de
comandos, por lo que en resumidas cuentas este proyecto consiste en hacer un frontend
web para una interfaz de línea de comandos.
La interfaz web no solo tendrá una sección de controles para manejar la sonda, sino que
también mostrará su estado de una manera gráfica (medidores de nivel, etc.) y dibujará
alguna gráfica sencilla (bytes recibidos vs. tiempo, etc.)


\section{\'Ambito}

Trabajo de Fin de Grado

 sondas de red de altas prestaciones

grupo de investigación \textit{High Performance Computing and Networking}. 

TODO: Ámbito del trabajo


\section{Motivación}

Necesaria simplificación gestión de sondas de red de altas prestaciones por medio de una interfaz gráfica

ampliar la funcionalidad ofreciendo monitorización 
TODO: Motivación del trabajo


\section{Objetivos}

Este Trabajo de Fin de Grado tiene los siguientes objetivos principales:

\begin{itemize}
  \item Desarrollar una interfaz web que permita la gestión de sondas de red de altas prestaciones de manera intuitiva y sencilla.
  Desarrollar una interfaz web que permita la gestión de sondas de red de altas prestaciones de manera intuitiva y sencilla.
  Desarrollar una interfaz web que permita la gestión de sondas de red de altas prestaciones de manera intuitiva y sencilla.

  \item Monitorizar el estado del servidor al que está conectada una sonda de red de altas prestaciones.
Desarrollar una interfaz web que permita la gestión de sondas de red de altas prestaciones de manera intuitiva y sencilla.
Desarrollar una interfaz web que permita la gestión de sondas de red de altas prestaciones de manera intuitiva y sencilla.

  \item Registrar estadísticas sobre la utilización de las sondas de red de altas prestaciones.
Desarrollar una interfaz web que permita la gestión de sondas de red de altas prestaciones de manera intuitiva y sencilla.
Desarrollar una interfaz web que permita la gestión de sondas de red de altas prestaciones de manera intuitiva y sencilla.
\end{itemize}

\section{Estructura del documento}

En el capítulo \ref{cap:estadoDelArte} se realiza un análisis del estado del arte. Se analizan tanto los sistemas de captura y reproducción de tráfico web existentes como las interfaces de gestión y monitorización de estos sistemas, para posteriormente extraer conclusiones sobre lo estudiado.

En el capítulo \ref{cap:defProyecto} se define la aplicación que se va a diseñar, así como la metodología seguida y las herramientas utilizadas en el proyecto.
En el capítulo \ref{cap:requisitos} se describen los requisitos funcionales y no funcionales de la aplicación.

En el capítulo \ref{cap:disenho} se formaliza el diseño de la aplicación a implementar, comentando la arquitectura de la aplicación y los módulos en los que se divide.
En el capítulo \ref{cap:implementacion} se documenta la implementación de la aplicación, estructurada en dos partes bien diferenciadas: \gls{back-end} y \gls{front-end}.
En el capítulo \ref{cap:pruebas} se explica el proceso de pruebas seguido para la verificación y validación de la aplicación construida, comprobando así el correcto funcionamiento de la misma.
En el capítulo \ref{cap:mantenimiento} se espeficica cómo se va a realizar el mantenimiento de la aplicación.

En el capítulo \ref{cap:conclusiones} se exponen las conclusiones finales sobre el trabajo realizado. Por último, en el capítulo \ref{cap:lineasDeTrabajoFuturo} se plantean posibles líneas de trabajo futuro que podrían ser abordadas con el objetivo de mejorar y ampliar diferentes aspectos de la aplicación desarrollada.